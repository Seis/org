\documentclass[12pt]{article}

\usepackage[english]{babel}
\usepackage[a4paper, top=1cm, bottom=1cm, left=2.5cm, right=2.5cm]{geometry}
\usepackage[utf8]{inputenc}
\usepackage[OT1]{fontenc}
\usepackage[datesep={}]{datetime2}
\makeatletter\let\saved@bibitem\@bibitem\makeatother
\usepackage[colorlinks, urlcolor={purple!75!black}, plainpages=true]{hyperref}
\makeatletter\let\@bibitem\saved@bibitem\makeatother
\usepackage[sfdefault]{quattrocento}
\usepackage{array, bibentry, ifthen, multirow, parskip, xcolor, pifont}

\newcommand*{\ruleline}[2]{%
    \makebox[\linewidth]{#1{\color{purple!75!black}
        \hspace{2pt}
        \hrulefill
        \hspace{2pt}
        \normalsize{\ding{111}}}}}
\newcommand*{\sep}{$\cdot\;$}

\newcommand*{\rulesection}[1]{\subsection*{\ruleline{#1}}\vspace{-1pt}}

\newenvironment{headertable}
  {\renewcommand{\arraystretch}{1.4}
   \newcolumntype{L}{>{\bf \raggedright}p{0.14\textwidth}}
   \newcolumntype{R}{p{0.82\textwidth}}
   \begin{tabular}{@{\hspace{0mm}}LR}}
  {\end{tabular}}

\newcommand*{\headerrow}[2]{#1 & #2 \\}

\newenvironment{contenttable}[1]
  {\rulesection{#1}
   \renewcommand{\arraystretch}{1.4}
   \newcolumntype{L}{p{0.82\textwidth}}
   \newcolumntype{R}{>{\raggedleft\arraybackslash}p{0.14\textwidth}}
   \begin{tabular}{@{\hspace{0mm}}LR}}
  {\end{tabular}}

\newcommand{\contentrow}[5]{
  \emph{#1} #2 & \multirow[t]{2}{58pt}{#3-- #4} \\
  \small{#5} & \\
}

\pagenumbering{gobble}

\begin{document}

\section*{\ruleline{Gustavo Zambonin}}

\begin{headertable}
  \headerrow{About \\[1ex] \scriptsize{(rev. \DTMtoday)}}{
    I am an information security consultant with 7+ years of experience and
    a solid academic background. I was a former lead of technical research and
    development for the Brazilian Digital Signature Standard. I specialize in
    quantum-safe cryptography and public-key infrastructures.
  }

  \headerrow{Address}{
    \href{https://zambonin.org}{\texttt{zambonin.org}}
    \sep \href{mailto:zambonin@pm.me}{\texttt{zambonin@pm.me}}
  }

  \headerrow{Languages}{
    Portuguese (native), English (fluent), French (beginner)
  }
\end{headertable}

\begin{contenttable}{Professional experience}
  \contentrow
    {Information security specialist}
    {
      in partnership with several institutions
    }
    {Sep/2017}
    {Today}
    {
      Some of my roles include acting as a consultant on digital signature
      standards; a ceremony operator deploying e-voting platforms;
      a quantum-safe blockchain researcher; and a computer forensic examiner
      measuring the accuracy of pictures from speed enforcement cameras.
      Maybe I can also help you, get in touch!
    }

  \contentrow
    {Technical lead and researcher}
    {
      at the Computer Security Lab of the Universidade Federal de Santa
      Catarina (UFSC)
    }
    {May/2016}
    {Feb/2024}
    {
      From 2020 onwards, I led the team whose job is to improve, maintain and
      add features to the Brazilian Digital Signature Standard official
      implementation, all derived applications, and normative documents. As
      a result, any Brazilian citizen is able to generate and verify digitally
      signed files per the latest standards.
      \vspace{1ex}

      Until 2019, as a software developer at that same team, I implemented
      several large new features to the signature verification service, such as
      a new responsive web interface, a REST API, and support for verification
      of CMS signatures.
    }
\end{contenttable}

\begin{contenttable}{Education}
  \contentrow
    {PhD in Computer Science}
    {
      at UFSC
    }
    {Mar/2024}
    {Today}
    {
      Currently researching novel combinatorial (un)ranking algorithms to
      generate random objects in quantum-safe cryptosystems.
    }
  \contentrow
    {MSc in Computer Science}
    {
      from UFSC (thesis:
      ``\href{https://repositorio.ufsc.br/handle/123456789/219497}{On the
      randomness of Rainbow signatures}'')
    }
    {Aug/2018}
    {Sep/2020}
    {
      I was a visiting researcher at Carleton University under
      a \href{http://archive.is/RHxm4}{Mitacs-CALAREO Globalink Research
      Award}, and a teaching assistant at UFSC that taught order theory,
      lattice theory and algebraic structures.
    }

  \contentrow
    {BSc in Computer Science}
    {
      from UFSC (thesis:
      ``\href{https://repositorio.ufsc.br/handle/123456789/187875}{Performance
      optimization for the Winternitz signature scheme}'')
    }
    {Mar/2013}
    {Jul/2018}
    {
      I was a teaching assistant for a probability and statistics class as
      a sophomore. Later, as a junior, I started working at the Computer
      Security Laboratory, developing features for the Brazilian Digital
      Signature Standard official implementation.
    }
\end{contenttable}

\rulesection{Personal values and interests}

I strive to solve problems and deliver elegant solutions with great efficiency,
attention to detail, and a minimal number of tools---most likely AWK, Bash,
tmux and Vim.

I'm also committed to bring out the best of the people working alongside me,
through frequent knowledge transfers and a constant feedback loop.

I'm enthusiastic about astronomy, immersive sim games, IBM keyboards
specifically older than myself and most songs with a saxophone line. 8)

\IfFileExists{\jobname.bib}{
\rulesection{Publications}

\nobibliography*

\bibentry{Zambonin:202412}.

\bibentry{Silvano:202412}.

\bibentry{Kamers:202412}.

\bibentry{Barbosa:202412}.

\bibentry{Martina:202412}.

\bibentry{Rosa:202504}.

\bibentry{Kamers:202409}.

\bibentry{Biage:202401}.

\bibentry{Mayr:202309}.

\bibentry{Perin:202106}.

\bibentry{Zambonin:201907}.

\bibentry{Perin:201806}.

\bibliographystyle{abbrv}
\nobibliography{\jobname}
}{}

\end{document}
