\documentclass[11pt]{article}

\usepackage{array, xcolor}
\usepackage[margin=1cm]{geometry}
\usepackage[utf8]{inputenc}
\usepackage{hyperref}
\hypersetup{
    colorlinks,
    linkcolor={red!50!black},
    citecolor={blue!50!black},
    urlcolor={blue!80!black}
}

\pagestyle{empty}

\newcommand \grayrule{\color{lightgray} \vrule}
\newcommand \last{\leftskip0.5cm $\ast$ }
\newcolumntype{L}{>{\raggedleft}p{0.18\textwidth}}
\newcolumntype{R}{p{0.79\textwidth}}

\begin{document}

\hspace{-0.9cm}
\begin{tabular*}{20.1cm}{l@{\extracolsep{\fill}}r}

\textbf{\Large Gustavo Zambonin}
& \texttt{+}55 (48) 9696 3133 \\
Av. Francisco Roberto da Silva, 925
& \texttt{gustavo.zambonin at grad.ufsc.br} \\
Biguaçu - SC (88160-284)
& \texttt{\href{http://github.com/zambonin}{github.com/zambonin}}
\end{tabular*}

\section*{Formação acadêmica}
\begin{tabular}{>{\bf}L!{\grayrule}R}

     2013 --
  atualmente & \textbf{Bacharelado em Ciência da Computação} \\
             & \last Universidade Federal de Santa Catarina,
             Florianópolis/SC \\
             & \last Previsão para conclusão: 2018 \\
\end{tabular}

\section*{Experiência profissional}
\begin{tabular}{>{\bf}L!{\grayrule}R}

2014 -- 2015 & \textbf{Monitoria} (Probabilidade e Estatística) \\
             & \last Análise exploratória de dados, cálculo de probabilidades
             de eventos, variáveis aleatórias discretas e contínuas,
             distribuições amostrais e estimação de parâmetros.
             \\ [10pt]

        2015 & \textbf{Monitoria} (Introdução à Ciência da Computação --
             voluntário) \\
             & \last Implementação de algoritmos em linguagens
             de programação (C, Java, Pascal, Python), estruturas de seleção e
             repetição, declaração e indexação de variáveis, compilação e
             execução de programas, entrada e saída de dados. \\
             & \last Ementa equivalente ou similar a Computação
             Científica I, Introdução à Computação, Programação Orientada a
             Objetos I e Introdução à Programação Orientada a Objetos.
\end{tabular}

\section*{Atividades extra-curriculares}
\begin{tabular}{>{\bf}L!{\grayrule}R}

        2015 & \textbf{Participante}: Minicurso de Linux (auxílio voluntário) \\
             & \last Organizado por PET Computação (6h) \\
             & \last Introdução ao funcionamento de sistemas operacionais do
             tipo GNU/Linux e suas diferenças, comandos básicos e gerenciamento
             de arquivos, redirecionamento de entrada e saída de comandos,
             segurança do sistema de arquivos, processos e programas em
             execução. \\ [10pt]

        2014 & \textbf{Ministrante}: A análise de dados usando o SEstatNet
             (Sistema de Ensino-Aprendizagem de Estatística via web) \\
             & \last 13ª SEPEX – Semana de Ensino, Pesquisa e Extensão da
             UFSC (4h) \\
             & \last Tópicos abordados: A filosofia da descrição e exploração
             de dados. Apresentação do SEstatNet. Descrição e exploração de
             dados uni e bi-variados usando o SEstatNet.
\end{tabular}

\section*{Qualificações}
\begin{tabular}{>{\bf}L!{\grayrule}R}

 Programação & \textbf{Nível intermediário}: \\
             & \last \texttt{bash}, C, C\texttt{++}, Google Sheets (fórmulas e
             JavaScript), Java, \LaTeX, \texttt{sh} \\
             & \last Python:
                \href{https://github.com/burnash/gspread}{gspread},
                \href{http://matplotlib.org/}{matplotlib},
                \href{http://www.numpy.org/}{NumPy},
                \href{http://docs.python-requests.org/en/latest/#}{Requests},
                \href{https://github.com/jmcarp/robobrowser}{RoboBrowser},
                \href{http://www.scipy.org/}{SciPy},
                \href{http://scrapy.org/}{Scrapy} \\
             & \textbf{Nível básico}: \\
             & \last AWK, \texttt{bc}, Haskell, Julia, Pascal, Prolog,
             \texttt{sed} \\ [10pt]

     Línguas & \textbf{Português (Brasil)} -- língua nativa \\
             & \textbf{Inglês} -- Proficiência operacional \\
             & \last CEFR B2 em \textit{listening
             comprehension} \\
             & \last CEFR C1 em \textit{structure and written
             expression} \\
             & \last CEFR C1 em \textit{reading
             comprehension} \\
\end{tabular}

\end{document}

% loosely based on:
% http://texblog.org/2012/04/25/writing-a-cv-in-latex/
