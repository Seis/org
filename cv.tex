% loosely based on: http://texblog.org/2012/04/25/writing-a-cv-in-latex/

\documentclass{article}

\usepackage[brazil]{babel}
\usepackage[a4paper, margin=1cm]{geometry}
\usepackage[colorlinks, urlcolor={blue!70!black}]{hyperref}
\usepackage[utf8]{inputenc}
\usepackage{array, xcolor}

\newenvironment{contenttable}[1]
  {\section*{#1}
   \newcolumntype{L}{>{\bf \raggedleft}p{0.13\textwidth}}
   \newcolumntype{R}{p{0.82\textwidth}}
   \begin{tabular}{L!{\color{lightgray} \vrule}R}}
  {\end{tabular}}

\newenvironment{smallitem}
  {\vspace{-2mm}\itemize%
   \setlength{\parskip}{0pt}
   \setlength{\itemsep}{2pt}}
  {\vspace{-2mm}\enditemize}


\pagenumbering{gobble}

\begin{document}

\begin{tabular}{@{\hspace{-5mm}}l}
  {\Large \textbf{Gustavo Zambonin}}    \\
  Av. Francisco Roberto da Silva, 925   \\
  Biguaçu/SC (88160--284)
\end{tabular}
\hfill
\begin{tabular}{r}
  \verb!+!55 (48) 99997 3940            \\
  \verb!gustavo.zambonin@grad.ufsc.br!  \\
  \url{https://github.com/zambonin}
\end{tabular}

\begin{contenttable}{Formação acadêmica}
  2013 --- atualmente & \textbf{Bacharelado em Ciência da Computação}
  \begin{smallitem}
    \item Universidade Federal de Santa Catarina, Florianópolis/SC
    \item Previsão para conclusão: 2018
    \item \url{http://lattes.cnpq.br/8192345791741876}
  \end{smallitem}
\end{contenttable}

\begin{contenttable}{Experiência profissional}
  Mar/2015 --- Ago/2015 & \textbf{Monitoria}
    (Introdução à Ciência da Computação --- voluntário)
  \begin{smallitem}
    \item Implementação de algoritmos em linguagens de programação (C,
      Java, Pascal, Python), estruturas de seleção e repetição, declaração e
      indexação de variáveis, compilação e execução de programas, entrada e
      saída de dados.
    \item Ementa similar a Computação Científica I, Introdução à
      Computação, Programação Orientada a Objetos I e Introdução à
      Programação Orientada a Objetos.
  \end{smallitem} \\

  Jul/2014 --- Ago/2015 & \textbf{Monitoria} (Probabilidade e Estatística)
  \begin{smallitem}
    \item Análise exploratória de dados, cálculo de probabilidades de
      eventos, variáveis aleatórias discretas e contínuas, distribuições
      amostrais e estimação de parâmetros.
  \end{smallitem}
\end{contenttable}

\begin{contenttable}{Atividades extracurriculares}
  Jul/2016 & \textbf{Participante}: \emph{Workshop in Combinatorics and
    Finite Fields in Cryptography and Codes}
  \begin{smallitem}
    \item Organizado por LabSEC/UFSC (16h) ---
      \url{http://cffcc.labsec.ufsc.br/}
    \item Tópicos abordados: criptografia, segurança, criptografia
      combinatorial, corpos finitos, design combinatorial, teoria
      de códigos, e áreas relacionadas.
  \end{smallitem} \\

  Abr/2015 & \textbf{Participante}: Minicurso de Linux (auxílio voluntário)
  \begin{smallitem}
    \item Organizado por PET Computação (6h) ---
      \url{http://pet.inf.ufsc.br/seccom/}
    \item Introdução ao funcionamento de sistemas operacionais do tipo
      GNU/Linux e suas diferenças, comandos básicos e gerenciamento de
      arquivos, redirecionamento de entrada e saída de comandos, segurança
      do sistema de arquivos, processos e programas em execução.
  \end{smallitem} \\

  Out/2014 & \textbf{Ministrante}: A análise de dados usando o SEstatNet
  \begin{smallitem}
    \item Sistema de Ensino-Aprendizagem de Estatística via web ---
      \url{http://sestatnet.ufsc.br/}
    \item 13ª SEPEX – Semana de Ensino, Pesquisa e Extensão da UFSC (4h)
    \item Tópicos abordados: A filosofia da descrição e exploração de
      dados. Apresentação do SEstatNet. Descrição e exploração de dados uni-
      e bi-variados usando o SEstatNet.
  \end{smallitem} \\

  Out/2013 & \textbf{Participante}: SECCOM {\small (Semana Acadêmica de
    Ciência da Computação e Sistemas de Informação)}
  \begin{smallitem}
    \item Organizado por PET Computação (10h)
    \item Palestras sobre temas diversos relacionados aos cursos.
  \end{smallitem}
\end{contenttable}

\begin{contenttable}{Qualificações}
  Tecnologias & \textbf{Experiência de 3+ anos com C, C++, Java,
    \LaTeX{}, Python}
  \begin{smallitem}
    \item Python: \href{https://github.com/burnash/gspread}{gspread},
      \href{http://matplotlib.org/}{matplotlib},
      \href{http://www.numpy.org/}{NumPy},
      \href{http://pyqt.sourceforge.net/Docs/PyQt5/}{PyQt},
      \href{http://docs.python-requests.org/}{Requests},
      \href{https://github.com/jmcarp/robobrowser}{RoboBrowser},
      \href{http://scrapy.org/}{Scrapy}
    \item Contato breve com Haskell, JavaScript, Julia, Pascal, Prolog
  \end{smallitem} \\

  & \textbf{Caixa de ferramentas}
  \begin{smallitem}
    \item 4+ anos com GNU/Linux (\texttt{bash}, \texttt{coreutils},
      \texttt{util-linux}, \texttt{awk}, \texttt{make}, \texttt{sed} etc.)
    \item GDB, Git, Google Sheets, Google Apps Script, QEMU, Vim, Valgrind
  \end{smallitem} \\

  Línguas & \textbf{Português (Brasil)} --- língua nativa \\
  & \textbf{Inglês} --- Proficiência operacional
  \begin{smallitem}
    \item CEFR B2 em \textit{listening comprehension}
    \item CEFR C1 em \textit{structure and written expression}
    \item CEFR C1 em \textit{reading comprehension}
  \end{smallitem}
\end{contenttable}

\end{document}
