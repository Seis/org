\documentclass[12pt]{article}

\usepackage[numbers]{natbib}
\usepackage{bibentry}
\usepackage[english]{babel}
\usepackage[a4paper, margin=1cm]{geometry}
\usepackage[utf8]{inputenc}
\usepackage[T1]{fontenc}
\usepackage[datesep={}]{datetime2}
\usepackage{amssymb, array, cmbright, enumitem, parskip, xcolor}
\usepackage[
    colorlinks, citecolor=black, urlcolor={gray!75!black}, plainpages=true
]{hyperref}

\newenvironment{sidetable}
  {\newcolumntype{L}{>{\bf \raggedright}p{0.14\textwidth}}
   \newcolumntype{R}{p{0.82\textwidth}}
   \begin{tabular}{@{\hspace{0mm}}LR}}
  {\end{tabular}}

\newenvironment{datetable}
  {\newcolumntype{R}{>{\raggedleft\arraybackslash}p{0.14\textwidth}}
   \newcolumntype{L}{p{0.82\textwidth}}
   \begin{tabular}{@{\hspace{0mm}}LR}}
  {\end{tabular}}

\newenvironment{contenttable}[1]
  {\subsection*{#1}
   \begin{datetable}}
  {\end{datetable}}

\newcommand{\csep}{$\;\cdot\;$}

\def\labelitemi{$\blacktriangleright$}

\pagenumbering{gobble}

\begin{document}

\nobibliography*

\begin{tabular}{@{\hspace{0mm}}l}
  {\Large \textbf{Gustavo Zambonin}} {\tiny rev. \DTMtoday}
\end{tabular}
\hfill \url{zambonin.org}
    \csep{} \texttt{\href{mailto:zambonin@pm.me}{zambonin@pm.me}}

\begin{sidetable}
  About & Quantum-safe cryptography researcher focusing on digital signature
    schemes, backed by years of contributions to the Brazilian Public-Key
    Infrastructure standards and a diversified set of projects related to
    information security.
\end{sidetable}

\begin{sidetable}
  Address & Laboratório de Segurança em Computação (LabSEC), INE 218, \\
          & Universidade Federal de Santa Catarina (UFSC), Florianópolis,
            88040-900, Brasil
\end{sidetable}

\begin{sidetable}
  Languages & Portuguese (native), English (fluent), French (beginner)
\end{sidetable}

\begin{contenttable}{Education}
  \emph{M.Sc. in Computer Science} (UFSC)
  \begin{itemize}
    \item Thesis: Reduction of key sizes on Rainbow-like
        multivariate signature schemes (to be defended on Jul/2020)
  \end{itemize} & Aug/2018--Today \\

  \emph{B.Sc. in Computer Science} (UFSC)
  \begin{itemize}
    \item \bibentry{Zambonin:201806}

  \end{itemize} & Mar/2013--Jul/2018 \\
\end{contenttable}

\begin{contenttable}{Academic activities}
  \emph{Visiting researcher} at Carleton University (Ottawa, Canada)
  \begin{itemize}
    \item Recipient of a \href{http://archive.is/RHxm4}{Mitacs-CALAREO
        Globalink Research Award} to study the security of Rainbow-like
          signature schemes
  \end{itemize} & Mar/2020--May/2020 \\

  \emph{Teaching assistance} for INE410134 - Post Quantum Cryptography and
    Computation
  \begin{itemize}
    \item Guest lecture and consultancy on multivariate cryptography to
        graduate students
  \end{itemize} & Aug/2019--Nov/2019 \\

  \emph{Co-supervision} of B.Sc. thesis
  \begin{itemize}
    \item \bibentry{Bittencourt:201911}
  \end{itemize} & Mar/2019--Dec/2019 \\

  \emph{Teaching assistance} for INE5601 - Mathematical Foundations of
    Informatics
  \begin{itemize}
    \item Classes on order theory, lattice theory, algebraic structures and
        group theory
  \end{itemize} & Aug/2018--Dec/2018 \\

  \emph{Lecturer} of ``Data analysis with
    \href{http://sestatnet.ufsc.br}{SEstatNet}'' on the 13th
    \href{https://sepex.ufsc.br/}{SEPEX} at UFSC
  \begin{itemize}
    \item Workshop on data analysis and processing with specialized tool
  \end{itemize} & Oct/2014 \\

  \emph{Teaching assistance} (undergraduate) for INE5405 - Probability and
    Statistics
  \begin{itemize}
    \item Consultancy on exploratory data analysis, probability distributions
        and events
  \end{itemize} & Aug/2014--Jul/2015 \\
\end{contenttable}

\begin{contenttable}{Publications}
  \hspace{0mm}\bibentry{Zambonin:201907} \\
  \bibentry{Perin:201806} \\
\end{contenttable}

\begin{contenttable}{Professional experience}
  \emph{Senior software developer} and systems administrator at LabSEC
  \begin{itemize}
    \item In partnership with the Brazilian National Institute of Information
        Technology (ITI). Major development effort towards the
          \href{https://verificador.iti.gov.br}{official digital signature
          validation tool} for the Brazilian Public-Key Infrastructure, that
          resulted in (i) a responsive new web interface; (ii) a clean API that
          enables headless/batch signature validation; (iii) enforced automated
          unit testing and continuous deployment practices.
  \end{itemize} & Jan/2018--Today \\

  \emph{Security ceremony agent} at LabSEC
  \begin{itemize}
    \item In partnership with public prosecutor's offices. Secure servers were
        provisioned to run online elections through the end-to-end verifiable
          voting system Helios, with reduced need for human-computer
          interaction.
  \end{itemize} & Oct/2018--Apr/2019 \\

  \emph{Researcher} of quantum-safe blockchain protocols at LabSEC
  \begin{itemize}
    \item In partnership with a novel blockchain platform. Co-developed a
        protocol to quantum-proof a blockchain, with secure substitution of
          wallets, replacement of cryptographic algorithms and zero downtime
          for the platform.
  \end{itemize} & Sep/2018--Mar/2019 \\

  \emph{Computer forensic examiner} at LabSEC
  \begin{itemize}
    \item In partnership with an intelligent transportation systems company. A
        complex data set was processed with native GNU/Linux tools and
          statistical techniques in order to verify the accuracy of pictures
          taken by speed enforcement cameras.
  \end{itemize} & Sep/2017--Apr/2018 \\

  \emph{Junior software developer} at LabSEC
  \begin{itemize}
    \item In partnership with a Brazilian digital security company. Developed a
        proof-of-concept signature validation module for PDF.js and a small
          library able to easily customize and instantiate most artifacts in a
          public-key infrastructure.
  \end{itemize} & Nov/2016--Dec/2017 \\

  \emph{Junior software developer} at LabSEC
  \begin{itemize}
    \item In partnership with the Brazilian National Institute of Information
        Technology (ITI). Implemented verification modules for CMS and PDF
          signatures in the official digital signature validation tool for the
          Brazilian Public-Key Infrastructure.
  \end{itemize} & May/2016--Oct/2016 \\
\end{contenttable}

\begin{contenttable}{Qualifications}
  \emph{Programming languages} and frameworks
  \begin{itemize}
    \item Worked with several Python frameworks: Flask, gspread,
        Helios, IPython, Matplotlib, NumPy, PyQt, Requests, robobrowser,
          Scrapy. For 5+ years routinely used AWK, Bash, C, C++, gnuplot, Java
          (JSE, JEE), \LaTeX{}, Make, SageMath, sed.
  \end{itemize} \\
  \emph{Software} and environment tools
  \begin{itemize}
    \item GNU/Linux exclusive user for 4+ years, with the following skill set:
        (i) text editors and IDEs include Vim, IntelliJ Idea, PyCharm; (ii)
          management software includes Git, GitLab CI/CD, Maven, Subversion;
          (iii) middleware includes Apache HTTP Server, Archiva, Tomcat,
          WildFly; (iv) miscellaneous software includes Clang Tools, Docker,
          GDB, OpenSSL, QEMU, PostgreSQL, SQLite, Valgrind.
  \end{itemize} \\
\end{contenttable}

\begin{contenttable}{Other interests}
  Enthusiastic about astronomy, the immersive sim game genre, IBM keyboards
    specifically older than the author and any song with a saxophone line. \\
\end{contenttable}

\bibliographystyle{abbrv}
\nobibliography{\jobname}

\end{document}
