\documentclass{article}

\usepackage[english]{babel}
\usepackage[a4paper, margin=1cm]{geometry}
\usepackage[utf8]{inputenc}
\usepackage[T1]{fontenc}
\usepackage[colorlinks, urlcolor={gray!75!black}]{hyperref}
\usepackage{amssymb, array, cmbright, parskip, qrcode, xcolor}

\newenvironment{contenttable}[1]
  {\section*{#1}
   \newcolumntype{L}{>{\bf \raggedleft}p{0.1\textwidth}}
   \newcolumntype{R}{p{0.82\textwidth}}
   \begin{tabular}{L!{\color{lightgray} \vrule}R}}
  {\end{tabular}}

\newcommand{\csep}{$\;\cdot\;$}
\newcommand{\lattesurl}{lattes.cnpq.br/8192345791741876}
\newcommand{\bscurl}{https://repositorio.ufsc.br/handle/123456789/187875}
\newcommand{\doi}{10.1109/ISCC.2018.8538642}

\def\labelitemi{$\blacktriangleright$}

\pagenumbering{gobble}

\begin{document}

\begin{tabular}{@{\hspace{0mm}}l}
  {\Large \textbf{Gustavo Zambonin}} \\
  \href{mailto:gustavo.zambonin@posgrad.ufsc.br}{gustavo.zambonin@posgrad.ufsc.br}
      \csep{} +55 48 999 973 940
      \csep{} \href{https://github.com/zambonin}{github.com/zambonin} \\
  925 Francisco Roberto da Silva Av. \csep{} 88160-284
      \csep{} Biguaçu, Santa Catarina \csep{} Brazil
\end{tabular}
\hfill \qrcode[height=1.35cm]{https://keybase.io/zambonin}

\begin{contenttable}{Education
    {\normalsize\normalfont (\href{http://\lattesurl}{\lattesurl})}}
  Today \\ Aug/2018 & \textbf{M.Sc. in Computer Science} (PPGCC/UFSC)
  \begin{itemize}
    \item Post-quantum cryptography researcher, in particular digital signature
        schemes
    \begin{itemize}
      \item Multivariate cryptography with focus on the Rainbow family
    \end{itemize}
    \item Teaching assistant for Mathematical Foundations of Informatics
    \begin{itemize}
      \item Order theory, lattice theory, algebraic structures, group theory
    \end{itemize}
  \end{itemize} \\

  Jul/2018 \\ Mar/2013 & \textbf{B.Sc. in Computer Science} (UFSC)
  \begin{itemize}
    \item Thesis named "\href{\bscurl}{Performance optimization for the
        Winternitz digital signature scheme}"
    \begin{itemize}
      \item Tuning the Winternitz hash-based digital signature scheme
          (\href{https://doi.org/\doi}{\doi})
    \end{itemize}
    \item Tutor for Probability and Statistics and Introduction to Computer
        Science
    \begin{itemize}
      \item Discrete and continuous probability distributions and their
          applications
      \item Building blocks of programming languages and GNU/Linux operating
          systems
    \end{itemize}
    \vspace{-5mm}
  \end{itemize}
\end{contenttable}

\begin{contenttable}{Professional experience}
  Today \\ May/2016 & \textbf{Computer Security Laboratory}
    (LabSEC/UFSC)
  \begin{itemize}
    \item Senior developer of the
        \href{https://verificador.iti.gov.br}{Conformance Verifier} for digital
          signatures in the Brazilian PKI
    \begin{itemize}
    \item Implementation of CMS and PDF signatures verification, and creation
        of a generic heuristic to classify digital signatures. Migration of
            the code base from Ant to Maven, unifying the package generation
            for Windows, macOS and Debian, and production of WAR files.
            Creation of the Web Verifier, developing Java servlets from
            scratch, with focus on maintainable code and ease of use.
            Employment of continuous integration and deployment practices using
            Docker, automating unit testing and package builds. Establishment
            of a validation environment updated frequently, using Apache HTTP
            Server and Tomcat. Migration of the code base history from
            Subversion to Git. Supervision of general development progress for
            CAdES, XAdES and PAdES signature verification.  Administration of
            all related virtual infrastructure.
    \end{itemize}
    \item Researcher of software requirements needed to support digital
        signatures in certain applications
    \begin{itemize}
      \item Description of a library model used to configure and instantiate a
          public-key infrastructure, with the intent of testing targeted
            software. Generated artifacts include multi-level certificate
            authorities, user certificates, certificate revocation lists and
            OCSP responses.
      \item Analysis of the effort needed to add digital signatures support on
          PDF.js, with the intent of enabling verification directly inside the
            browser. Creation of fork that verifies a small set of digital
            signatures. Description of a software model that allows
            modularization and decoupling of the digital signature verification
            component from PDF.js.
    \end{itemize}
    \item Various roles in other projects related to information security
    \begin{itemize}
      \item Researcher of a protocol to quantum-proof a blockchain, with
          secure substitution of wallets, replacement of cryptographic
            algorithms and zero downtime for the platform.
      \item Operator of security ceremonies, in which secure servers were
          provisioned to run verifiable online elections through Helios, itself
            based on frameworks such as Celery and Django.
      \item Forensic computer examiner of images generated by speed enforcement
          cameras, handling large amounts of heterogeneous data with tools such
            as \texttt{coreutils}, with the intent of proving accuracy and
            integrity of such files.
    \end{itemize}
    \vspace{-5mm}
  \end{itemize}
\end{contenttable}

\begin{contenttable}{Qualifications}
  Coding & AWK, Bash, C, C++, Java (JSE, JEE), \LaTeX{}, sed

           Python (Flask, gspread, Helios, IPython, Matplotlib, NumPy, PyQt,
           Requests, robobrowser, Scrapy)
  \begin{itemize}
    \item Brief experience: CSS, gnuplot, Haskell, JavaScript, Julia, Octave,
        Pascal, Prolog, SageMath
    \vspace{-4mm}
  \end{itemize} \\

  Environs & GNU/Linux, Vim, IntelliJ Idea, PyCharm, Eclipse, Visual Studio
    Code
  \begin{itemize}
    \item Management: Ant, Git, GitLab CI/CD, Make, Maven, Subversion
    \vspace{-4mm}
  \end{itemize} \\

  Software & Clang Static Analyzer, Docker, GDB, QEMU, PostgreSQL, SQLite,
    Valgrind
  \begin{itemize}
    \item Middleware: Apache HTTP Server, Archiva, Tomcat, WildFly
    \vspace{-4mm}
  \end{itemize} \\

  Languages & Portuguese (native), English (fluent), French (beginner)
\end{contenttable}

\end{document}
