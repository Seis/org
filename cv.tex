\documentclass[12pt]{article}

\usepackage[numbers]{natbib}
\usepackage{bibentry}
\usepackage[english]{babel}
\usepackage[a4paper, margin=1cm]{geometry}
\usepackage[utf8]{inputenc}
\usepackage[T1]{fontenc}
\usepackage[datesep={}]{datetime2}
\usepackage{amssymb, array, cmbright, enumitem, parskip, xcolor}
\usepackage[
    colorlinks, citecolor=black, urlcolor={gray!75!black}, plainpages=true
]{hyperref}

\newenvironment{sidetable}
  {\newcolumntype{L}{>{\bf \raggedright}p{0.14\textwidth}}
   \newcolumntype{R}{p{0.82\textwidth}}
   \begin{tabular}{@{\hspace{0mm}}LR}}
  {\end{tabular}}

\newenvironment{datetable}
  {\newcolumntype{R}{>{\raggedleft\arraybackslash}p{0.14\textwidth}}
   \newcolumntype{L}{p{0.82\textwidth}}
   \begin{tabular}{@{\hspace{0mm}}LR}}
  {\end{tabular}}

\newenvironment{contenttable}[1]
  {\subsection*{#1}
   \begin{datetable}}
  {\end{datetable}}

\newcommand{\csep}{$\;\cdot\;$}

\def\labelitemi{$\blacktriangleright$}

\pagenumbering{gobble}

\begin{document}

\nobibliography*

\begin{tabular}{@{\hspace{0mm}}l}
  {\Large \textbf{Gustavo Zambonin}} {\tiny rev. \DTMtoday}
\end{tabular}
\hfill \href{https://zambonin.org}{\texttt{zambonin.org}}
    \csep{} \href{mailto:zambonin@pm.me}{\texttt{zambonin@pm.me}}

\begin{sidetable}
  About & Quantum-safe cryptography researcher focusing on digital signature
    schemes, backed by years of contributions to the Brazilian Public-Key
    Infrastructure standards and a diversified set of projects related to
    information security.
\end{sidetable}

\begin{sidetable}
  Address & Laboratório de Segurança em Computação (LabSEC), INE 218, \\
          & Universidade Federal de Santa Catarina (UFSC), Florianópolis,
            88040-900, Brasil
\end{sidetable}

\begin{sidetable}
  Languages & Portuguese (native), English (fluent), French (beginner)
\end{sidetable}

\begin{contenttable}{Education}
  \emph{M.Sc. in Computer Science} (UFSC)
  \begin{itemize}
    \item \bibentry{Zambonin:202009}
  \end{itemize} & Aug/2018--Sep/2020 \\

  \emph{B.Sc. in Computer Science} (UFSC)
  \begin{itemize}
    \item \bibentry{Zambonin:201806}

  \end{itemize} & Mar/2013--Jul/2018 \\
\end{contenttable}

\begin{contenttable}{Academic activities}
  \emph{Visiting researcher} at Carleton University (Ottawa, Canada)
  \begin{itemize}
    \item Recipient of a \href{http://archive.is/RHxm4}{Mitacs-CALAREO
        Globalink Research Award} to study the security of Rainbow-like
          signature schemes
  \end{itemize} & Mar/2020--Jun/2020 \\

  \emph{Teaching assistance} for INE410134 - Post Quantum Cryptography and
    Computation
  \begin{itemize}
    \item Guest lecture and consultancy on multivariate cryptography to
        graduate students
  \end{itemize} & Aug/2019--Nov/2019 \\

  \emph{Co-supervision} of B.Sc. thesis
  \begin{itemize}
    \item \bibentry{Bittencourt:201911}
  \end{itemize} & Mar/2019--Dec/2019 \\

  \emph{Teaching assistance} for INE5601 - Mathematical Foundations of
    Informatics
  \begin{itemize}
    \item Classes on order theory, lattice theory, algebraic structures and
        group theory
  \end{itemize} & Aug/2018--Dec/2018 \\

  \emph{Lecturer} of ``Data analysis with
    \href{http://sestatnet.ufsc.br}{SEstatNet}'' on the 13th
    \href{https://sepex.ufsc.br/}{SEPEX} at UFSC
  \begin{itemize}
    \item Workshop on data analysis and processing with specialized tool
  \end{itemize} & Oct/2014 \\

  \emph{Teaching assistance} (undergraduate) for INE5405 - Probability and
    Statistics
  \begin{itemize}
    \item Consultancy on exploratory data analysis, probability distributions
        and events
  \end{itemize} & Aug/2014--Jul/2015 \\
\end{contenttable}

\begin{contenttable}{Publications}
  \hspace{0mm}\bibentry{Zambonin:201907} \\
  \bibentry{Perin:201806} \\
\end{contenttable}

\begin{contenttable}{Professional experience}
  \emph{Software project manager} at LabSEC
  \begin{itemize}
    \item In partnership with the Brazilian National Institute of Information
        Technology (ITI). Coordinates the development of desktop, web and
          mobile tools used in the Brazilian Public-Key Infrastructure
          (ICP-Brasil) to generate and validate digital signatures.
  \end{itemize} & Jan/2020--Today \\

  \emph{Security ceremony agent} at LabSEC
  \begin{itemize}
    \item In partnership with several public institutions. Secure servers are
        provisioned to run online elections through the end-to-end verifiable
          voting system Helios, with reduced need for human-computer
          interaction.
  \end{itemize} & Oct/2018--Apr/2019 \\

  \emph{Senior software developer} and systems administrator at LabSEC
  \begin{itemize}
    \item In partnership with ITI. Major development effort towards the
          \href{https://verificador.iti.gov.br}{official digital signature
          verification tool} of ICP-Brasil, that resulted in a responsive new
          web interface, an API that enables headless/batch signature
          verification, enforced automated unit testing and continuous
          deployment practices.
  \end{itemize} & Jan/2018--Dec/2019 \\

  \emph{Researcher} of quantum-safe blockchain protocols at LabSEC
  \begin{itemize}
    \item In partnership with a novel blockchain platform. Co-developed a
        protocol to quantum-proof a blockchain, with secure substitution of
          wallets, replacement of cryptographic algorithms and zero downtime
          for the platform.
  \end{itemize} & Sep/2018--Mar/2019 \\

  % \emph{Computer forensic examiner} at LabSEC
  % \begin{itemize}
  %   \item In partnership with an intelligent transportation systems company. A
  %       complex data set was processed with native GNU/Linux tools and
  %         statistical techniques in order to verify the accuracy of pictures
  %         taken by speed enforcement cameras.
  % \end{itemize} & Sep/2017--Apr/2018 \\

  \emph{Junior software developer} at LabSEC
  \begin{itemize}
    \item In partnership with a Brazilian digital security company. Developed a
        proof-of-concept signature verification module for PDF.js and
          a customizable library to create artifacts in a
          public-key infrastructure.
  \end{itemize} & Nov/2016--Dec/2017 \\

  \emph{Junior software developer} at LabSEC
  \begin{itemize}
    \item In partnership with ITI. Implemented support for CMS signatures
        (standalone or embedded in PDFs) in the official digital signature
          verification tool of ICP-Brasil.
  \end{itemize} & May/2016--Oct/2016 \\
\end{contenttable}

\begin{contenttable}{Qualifications}
  \emph{Programming languages} and frameworks
  \begin{itemize}
    \item Worked with several Python frameworks: Flask, gspread,
        Helios, IPython, Matplotlib, NumPy, PyQt, Requests, robobrowser,
          Scrapy. For 5+ years routinely used AWK, Bash, C, C++, gnuplot, Java
          (JSE, JEE), \LaTeX{}, Make, SageMath, sed.
  \end{itemize} \\
  \emph{Software} and environment tools
  \begin{itemize}
    \item GNU/Linux exclusive user for 4+ years, with the following skill set:
        (i) text editors and IDEs include Vim, IntelliJ Idea, PyCharm; (ii)
          management software includes Git, GitLab CI/CD, Maven, Subversion;
          (iii) middleware includes Apache HTTP Server, Archiva, Tomcat,
          WildFly; (iv) miscellaneous software includes Clang Tools, Docker,
          GDB, OpenSSL, QEMU, PostgreSQL, SQLite, Valgrind.
  \end{itemize} \\
\end{contenttable}

\begin{contenttable}{Other interests}
  Enthusiastic about astronomy, the immersive sim game genre, IBM keyboards
    specifically older than the author and any song with a saxophone line. \\
\end{contenttable}

\bibliographystyle{abbrv}
\nobibliography{\jobname}

\end{document}
