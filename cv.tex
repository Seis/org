\documentclass[12pt]{article}

\usepackage[english]{babel}
\usepackage[a4paper, margin=2cm]{geometry}
\usepackage[utf8]{inputenc}
\usepackage[T1]{fontenc}
\usepackage[datesep={}]{datetime2}
\usepackage[colorlinks, urlcolor={gray!75!black}, plainpages=true]{hyperref}
\usepackage{array, cmbright, parskip, xcolor}

\newenvironment{sidetable}
  {\renewcommand{\arraystretch}{1.4}
   \newcolumntype{L}{>{\bf \raggedright}p{0.14\textwidth}}
   \newcolumntype{R}{p{0.82\textwidth}}
   \begin{tabular}{@{\hspace{0mm}}LR}}
  {\end{tabular}}

\newenvironment{contenttable}[1]
  {\subsection*{#1}
   \renewcommand{\arraystretch}{1.4}
   \newcolumntype{L}{p{0.82\textwidth}}
   \newcolumntype{R}{>{\raggedleft\arraybackslash}p{0.14\textwidth}}
   \begin{tabular}{@{\hspace{0mm}}LR}}
  {\end{tabular}}

\pagenumbering{gobble}

\begin{document}

{\Large \textbf{Gustavo Zambonin}} {\tiny rev. \DTMtoday}
\hfill \href{https://zambonin.org}{\texttt{zambonin.org}}
    $\;\cdot\;$ \href{mailto:zambonin@pm.me}{\texttt{zambonin@pm.me}}

\begin{sidetable}
  About     & Quantum-safe cryptography researcher focusing on digital
              signature schemes, backed by years of contributions to the
              Brazilian Public-Key Infrastructure standards and a diversified
              set of projects related to information security. \\

  Address   & Laboratório de Segurança em Computação (LabSEC), INE 218,
              Universidade Federal de Santa Catarina (UFSC), Florianópolis,
              88040-900, Brasil \\

  Languages & Portuguese (native), English (fluent), French (beginner)
\end{sidetable}

\begin{contenttable}{Education}
  \emph{PhD in Computer Science} at the University of Ottawa (currently
    researching post-quantum cryptography and combinatorial cryptography)
    & Sep/2020--Today \\

  \emph{MSc in Computer Science} from the Universidade Federal de Santa
    Catarina (thesis named
    \href{https://repositorio.ufsc.br/handle/123456789/219497}{On the
    randomness of Rainbow signatures})
    & Aug/2018--Sep/2020 \\

  \emph{BSc in Computer Science} from the Universidade Federal de Santa
    Catarina (thesis named
    \href{https://repositorio.ufsc.br/handle/123456789/187875}{Performance
    optimization for the Winternitz signature scheme})
    & Mar/2013--Jul/2018
\end{contenttable}

\begin{contenttable}{Academic activities}
  \emph{Visiting researcher} at Carleton University (Canada) -- Recipient of
    a \href{http://archive.is/RHxm4}{Mitacs-CALAREO Globalink Research Award}
    to study the security of Rainbow-like signature schemes
    & Mar/2020--Jun/2020 \\

  \emph{Teaching assistant} for INE5601 - Mathematical Foundations of
    Informatics at UFSC (Classes on order theory, lattice theory, algebraic
    structures and group theory)
    & Aug/2018--Dec/2018 \\

  \emph{Teaching assistant} for INE5405 - Probability and Statistics at UFSC
    (Consultancy on exploratory data analysis, probability distributions and
    events)
    & Aug/2014--Jul/2015
\end{contenttable}

\begin{contenttable}{Professional experience}
  \emph{Senior developer and project manager} at LabSEC -- In partnership with
    the Brazilian National Institute of Information Technology, coordinates the
    development of desktop, web and mobile tools used in the Brazilian
    Public-Key Infrastructure to generate and validate digital signatures.
    & May/2016--Today \\

    Has also worked as a (i) ceremony operator deploying Helios-based e-voting
    platforms; (ii) quantum-safe blockchain researcher; (iii) computer forensic
    examiner measuring the accuracy of pictures from speed enforcement cameras.
\end{contenttable}

\begin{contenttable}{Qualifications and other interests}
  For 5+ years routinely uses AWK, Bash, C, C++, gnuplot, Java (JSE, JEE),
    \LaTeX{}, Make, Python, SageMath, sed on a GNU/Linux distribution. \\

  Enthusiastic about astronomy, the immersive sim game genre, IBM keyboards
    specifically older than the author and any song with a saxophone line.
\end{contenttable}

\end{document}
