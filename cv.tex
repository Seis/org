% loosely based on:
% http://texblog.org/2012/04/25/writing-a-cv-in-latex/

\documentclass[11pt]{article}

\usepackage{array, xcolor}
\usepackage[brazil]{babel}
\usepackage[margin=1cm]{geometry}
\usepackage[utf8]{inputenc}
\usepackage[colorlinks, urlcolor={blue!80!black}]{hyperref}

\pagestyle{empty}

\newenvironment{headertable}{
    \hspace{-0.9cm}
    \begin{tabular*}{20.1cm}{l@{\extracolsep{\fill}}r}
}{
    \end{tabular*}
}

\newenvironment{contenttable}{
    \newcolumntype{L}{>{\bf \raggedleft}p{0.13\textwidth}}
    \newcolumntype{R}{p{0.84\textwidth}}
    \begin{tabular}{L!{\color{lightgray} \vrule}R}
}{
    \end{tabular}
}

\newenvironment{smallitem}{
    \vspace{-2mm}
    \begin{itemize}
    \setlength{\parskip}{0pt}
    \setlength{\itemsep}{2pt}
}{
    \vspace{-4mm} \end{itemize}
}

\newcommand{\customsec}[1]{\vspace{-4mm} \section*{#1}}

\begin{document}

\begin{headertable}
\bf \Large Gustavo Zambonin         & \texttt{+}55 (48) 9696 3133            \\
Av. Francisco Roberto da Silva, 925 & \texttt{gustavo.zambonin@grad.ufsc.br} \\
Biguaçu/SC (88160-284)              & \url{https://github.com/zambonin}      \\
\end{headertable}

\customsec{Formação acadêmica}
\begin{contenttable}
    2013 -- atualmente & \textbf{Bacharelado em Ciência da Computação}
    \begin{smallitem}
        \item Universidade Federal de Santa Catarina, Florianópolis/SC
        \item Previsão para conclusão: 2018
    \end{smallitem}
\end{contenttable}

\customsec{Experiência profissional}
\begin{contenttable}
    2014 -- 2015 & \textbf{Monitoria} (Probabilidade e Estatística)
    \begin{smallitem}
        \item Análise exploratória de dados, cálculo de probabilidades de
        eventos, variáveis aleatórias discretas e contínuas, distribuições
        amostrais e estimação de parâmetros.
    \end{smallitem} \\

    2015 & \textbf{Monitoria} (Introdução à Ciência da Computação -- voluntário)
    \begin{smallitem}
        \item Implementação de algoritmos em linguagens de programação (C,
        Java, Pascal, Python), estruturas de seleção e repetição, declaração e
        indexação de variáveis, compilação e execução de programas, entrada e
        saída de dados.
        \item Ementa similar a Computação Científica I, Introdução à
        Computação, Programação Orientada a Objetos I e Introdução à
        Programação Orientada a Objetos.
    \end{smallitem}
\end{contenttable}

\customsec{Atividades extracurriculares}
\begin{contenttable}
    2015 & \textbf{Participante}: Minicurso de Linux (auxílio voluntário)
    \begin{smallitem}
        \item Organizado por PET Computação (6h)
        \item Introdução ao funcionamento de sistemas operacionais do tipo
        GNU/Linux e suas diferenças, comandos básicos e gerenciamento de
        arquivos, redirecionamento de entrada e saída de comandos, segurança do
        sistema de arquivos, processos e programas em execução.
    \end{smallitem} \\

    2014 & \textbf{Ministrante}: A análise de dados usando o SEstatNet (Sistema
    de Ensino-Aprendizagem de Estatística via web)
    \begin{smallitem}
        \item 13ª SEPEX – Semana de Ensino, Pesquisa e Extensão da UFSC (4h)
        \item Tópicos abordados: A filosofia da descrição e exploração de
        dados. Apresentação do SEstatNet. Descrição e exploração de dados uni-
        e bi-variados usando o SEstatNet.
    \end{smallitem}
\end{contenttable}

\customsec{Qualificações}
\begin{contenttable}
    Programação & \textbf{Nível intermediário}:
    \begin{smallitem}
        \item \texttt{bash}, C, C\texttt{++}, Google Sheets (fórmulas e
        JavaScript), Java, \LaTeX, \texttt{sh}
        \item Python: \href{https://github.com/burnash/gspread}{gspread},
        \href{http://matplotlib.org/}{matplotlib},
        \href{http://www.numpy.org/}{NumPy},
        \href{http://docs.python-requests.org/en/latest/#}{Requests},
        \href{https://github.com/jmcarp/robobrowser}{RoboBrowser},
        \href{http://www.scipy.org/}{SciPy},
        \href{http://scrapy.org/}{Scrapy}
    \end{smallitem} \\

    & \textbf{Nível básico}:
    \begin{smallitem}
        \item AWK, \texttt{bc}, Haskell, Julia, Pascal, Prolog, \texttt{sed}
    \end{smallitem} \\

    Línguas & \textbf{Português (Brasil)} -- língua nativa \\
    & \textbf{Inglês} -- Proficiência operacional
    \begin{smallitem}
        \item CEFR B2 em \textit{listening comprehension}
        \item CEFR C1 em \textit{structure and written expression}
        \item CEFR C1 em \textit{reading comprehension}
    \end{smallitem}
\end{contenttable}

\end{document}
